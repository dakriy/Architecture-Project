\documentclass[a4paper]{article}

%% Language and font encodings
\usepackage[english]{babel}
\usepackage[utf8x]{inputenc}
\usepackage[T1]{fontenc}

%% Sets page size and margins
\usepackage[a4paper,top=3cm,bottom=2cm,left=3cm,right=3cm,marginparwidth=1.75cm]{geometry}

%% Useful packages
\usepackage{amsmath}
\usepackage{graphicx}
\graphicspath{ {images/} }
\usepackage[colorinlistoftodos]{todonotes}
\usepackage[colorlinks=true, allcolors=blue]{hyperref}
\usepackage{minted} %vhdl syntax highlighting

\title{Project Documentation}
\author{Jacob Priddy, Andrew Glencross, Charles Oroko, Seth Ballance}

\begin{document}
\maketitle

\section{Introduction}
\paragraph{}
Our processor implementation and instruction set are very lightweight meant to be flashed to and run an FPGA. This is an experimental project for a class and is not recommended for use in military or medical applications.

\section{Processor}
\paragraph{}
The processor does not have any floating point capabilities. There are 16 registers (including the program counter). It has 4KB of data memory and 64KB of program memory. It is pipelined, with 4 stages. All numbers are treated as two's compliment by the processor. Word size is 16 bits or 2 bytes stored big endian. 

\section{Instruction Set}
\paragraph{}
All instructions are 1 word. Registers are addressed with 4 bits. 0000 would reference the PC while 0001 references R1. If there are 2 operands, the result gets put into the first operand. Opcode is 4 bits. All immediate values are interpreted to be decimal values.


\paragraph{}
Here is the complete instruction list. 

\begin{itemize}
\item R - R type instruction
\item I - I type instruction
\item J - J type instruction
\item P - Pseudo instruction
\end{itemize}

\subsection{Instruction List}

\begin{center}
\begin{tabular}{|c | c | c |}
\hline
Instruction Description & Instruction & Type\\ \hline
add two registers& 					add & R\\ \hline
add an immediate to a register& 	addi& I\\ \hline
shift a register right arithmetic& 	sra & I\\ \hline
shift a register right logical&		srl & I\\ \hline
shift a register left&				sl  & I\\ \hline
bitwise or two registers&			or  & R\\ \hline
bitwise or a register and immediate&ori & I\\ \hline
bitwise and two registers&			and & R\\ \hline
bitwise and a register and immediate&andi&I\\ \hline
invert bits in a register&			not& I\\ \hline
jump on register zero&				jz 	& J\\ \hline
unconditional jump&					j	& J\\ \hline
load immediate into a register&		loadi&I\\ \hline
move words from or to memory, or other registers&mov&R\\ \hline
No operation&						nop	& P\\ \hline

\end{tabular}
\end{center}

\subsection{Register List}
\begin{center}
\begin{tabular}{| c | c | c |}
\hline
Register Name & Use & Binary Representation\\ \hline
R0 & General purpose register & 0000\\ \hline
R1 & General purpose register & 0001\\ \hline
R2 & General purpose register & 0010\\ \hline
R3 & General purpose register & 0011\\ \hline
R4 & General purpose register & 0100\\ \hline
R5 & General purpose register & 0101\\ \hline
R6 & General purpose register & 0110\\ \hline
R7 & General purpose register & 0111\\ \hline
R8 & General purpose register & 1000\\ \hline
R9 & General purpose register & 1001\\ \hline
R10 & General purpose register& 1010\\ \hline
R11 & General purpose register& 1011\\ \hline
R12 & General purpose register& 1100\\ \hline
R13 & General purpose register& 1101\\ \hline
R14 & General purpose register& 1110\\ \hline
R15 & General purpose register& 1111\\ \hline
\end{tabular}
\end{center}

\section{Nomenclature}
Bit key:
\begin{center}
\begin{tabular}{| c | c |}
\hline
Bit & Definition\\ \hline
o & opcode bit\\ \hline
s & source/destination register bit \\ \hline
t & temporary register bit\\ \hline
i & immediate value\\ \hline
m & address mode bit\\ \hline
x & unused/reserved bit \\ \hline
\end{tabular}
\end{center}

\section{Syntax}
\subsection{General Syntax}
\paragraph{}
Instruction identifiers are to be followed by their arguments. If there are two arguments, they are to be separated by a comma. Only 1 instruction per line is allowed.

\subsection{Comments}
\paragraph{}
Comments start with a semicolon ';'. Anything after a semicolon to the end of the line is ignored by the assembler no matter where it is on a line.

\subsection{Labels}
\paragraph{}
Labels are to be followed by a colon ':', and may not contain spaces. Labels must come first in the line. They may be on a line by themselves or on a line with an instruction.


\section{R-Type Instructions}

\subsection{Description}

\begin{itemize}
\item 4 bit opcode
\item 4 bits for operand 1
\item 4 bits for operand 2
\item 2 bits to select address mode (only used for mov instructions)
\item 2 unused bits
\end{itemize}

\subsection{Bit Field}

\begin{center}
\begin{tabular}{| c | c | c | c | c | c | c | c | c | c | c | c | c | c | c | c |}
\hline
o&o&o&o&s&s&s&s&t&t&t&t&m&m&x&x\\ \hline
\end{tabular}
\end{center}

\subsection{List}
\begin{center}
\begin{tabular}{| c | c | c | c |}
\hline
Instruction & description & opcode & parameters \\ \hline
add & Add two registers & 0000 & 2 reg \\ \hline
and & And two registers & 0001 & 2 reg \\ \hline
or & Or two registers & 0010 & 2 reg \\ \hline
mov & Moves first register to second register & 0011 & 2 reg \\ \hline
\end{tabular}
\end{center}

\subsection{Comments}
\subsection{add}
\paragraph{} Adds the first register and the second register, and stores the result into the first register. Treats numbers as two's compliment. 
\paragraph{Example:} 
\subparagraph{add r1, r2} r1 = r1 + r2
\subparagraph{add PC, r15} PC = PC + r15

\subsection{and}
\paragraph{} Does a bitwise AND operation on the bits in s and t, and store the results in s. 
\paragraph{Example:} 
\subparagraph{and r1, r2} r1 = r1 \& r2

\subsection{or}
\paragraph{} Does a bitwise OR operation on the bits in s and t, and store the results in s. 
\paragraph{Example:}
\subparagraph{or r1, r2} r1 = r1 | r2

\subsubsection{mov}
\paragraph{} The mov instruction is what is used to retrieve and write to memory. If you surround the register identifier in parentheses, the processor will treat the number in the register as a memory address, and read/write to/from said address.\\

\paragraph{Example:}
\subparagraph{mov r2, (r3)} r2 = address(r3)

will copy the word at the address stored in r3 into r2

\subparagraph{mov (r14), (r15)} address(r14) = address(r15)

will copy the word at the address stored in r15 into the word pointed to by the address stored in r14.


\section{I-Type Instructions}
\subsection{Description}
\begin{itemize}
\item 4 bit opcode
\item 4 bits for reg 1
\item 8 bits for immediate value
\end{itemize}

\subsection{Bit Field}
\begin{center}
\begin{tabular}{| c | c | c | c | c | c | c | c | c | c | c | c | c | c | c | c |}
\hline
o&o&o&o&s&s&s&s&i&i&i&i&i&i&i&i\\ \hline
\end{tabular}
\end{center}


\subsection{List}
\begin{center}
\begin{tabular}{| c | c | c | c |}
\hline
Instruction & description & opcode & parameters \\ \hline
srl & shift right logical & 0100 & 1 reg 1 immediate \\ \hline
sra & shift right arithmetic & 0101 & 1 reg 1 immediate \\ \hline
sl & shift left arithmetic & 0110 & 1 reg 1 immediate \\ \hline
not & flip bits in a register & 0111 & 1 reg 1 immediate \\ \hline
andi & Does the and operation on a register and an immedate & 1000 & 1 reg 1 immediate \\ \hline
addi & Add an immediate value to a register & 1001 & 1 reg 1 immediate \\ \hline
ori & Does the or operation on a register and an immedate & 1010 & 1 reg 1 immediate \\ \hline
loadi & Loads an immedate value into a register & 1011 & 1 reg 1 immedate \\ \hline
\end{tabular}
\end{center}

\subsection{Comments}
\subsubsection{srl}
\paragraph{} Shifts the number in the register to the right <immediate> number of times. Makes the incoming bit 0. Will not accept values > 16
\paragraph{Example:} 
\subparagraph{srl r1, 10} r1 = r1 >> 10

\subsubsection{sra}
\paragraph{} Shifts the number in the register to the right <immediate> number of times. Keeps the sign of the number. Will not accept values > 16
\paragraph{Example:}
\subparagraph{sra r1, 11} r1 = r1 >> 11

\subsubsection{sl}
\paragraph{} Shifts the number in the register to the left <immediate> number of times. Makes the incoming bit 0. Will not accept values > 16
\paragraph{Example:} 
\subparagraph{sl r1, 3} r1 = r1 << 3

\subsubsection{not}
\paragraph{} Inverts the bits in a register up to <immediate> starting from the least significant bits. If no immediate value is provided, whole register will be flipped.
\paragraph{Example:}
\subparagraph{not r1} r1 = \~r1
\subparagraph{not r1, 2} r1 = r1 except last 2 bits (least significant) will be inverted

\subsubsection{andi}
\paragraph{} Ands an immediate with a register, and stores the result into the register. Can only and values from -127 to 127. If larger values are needed, see the R type instruction.
\paragraph{Example:}
\subparagraph{andi r1, 123} r1 = r1 \& 0x7B

\subsubsection{addi}
\paragraph{} Adds an immediate with a register, and stores the result into the register. Can only add an immediate value from -127 to 127. If larger values are needed, see the R type instruction.
\paragraph{Example:}
\subparagraph{addi r1, 123} r1 = r1 + 0x7B

\subsubsection{ori}
\paragraph{} Bitwise or's  an immediate with a register, and stores the result into the register. Can only or values from -127 to 127. If larger values are needed, see the R type instruction.
\paragraph{Example:}
\subparagraph{ori r1, 123} r1 = r1 | 0x7B

\subsubsection{loadi}
\paragraph{} Stores the immediate value into the specified register. Can only load in values from -127 to 127.
\paragraph{Example:}
\subparagraph{loadi r5, 23} r1 = r1 \& 0x17

\section{J-Type}
\subsection{Description}
\begin{itemize}
\item 4 bit opcode
\item 4 bits for comparison register
\item 8 bits for jump address
\end{itemize}

\subsection{List}
\begin{center}
\begin{tabular}{| c | c | c | c |}
\hline
Instruction & description & opcode & parameters \\ \hline
jz & Jump to an offset if a register is zero & 1100 & 1 reg 1 imm \\ \hline
j & Jump unconditionally & 1101 & immediate \\ \hline
\end{tabular}
\end{center}

\subsection{Comments}
\subsubsection{General}
\paragraph{Warning:} it is possible to get the processor stuck by jumping a jump instruction.

\subsubsection{jz}
\paragraph{} Jumps to a label. Stores the offset, so it can only be used to jump to an instruction $\pm 127$ instructions away. If you need to go further, use the j instruction.

\paragraph{Bit Field}
\begin{center}
\begin{tabular}{| c | c | c | c | c | c | c | c | c | c | c | c | c | c | c | c |}
\hline
o&o&o&o&s&s&s&s&i&i&i&i&i&i&i&i\\ \hline
\end{tabular}
\end{center}

\paragraph{Example:}
\subparagraph{jz r1, label} jumps to label if the register is 0


\subsubsection{j}
\paragraph{} Jumps to that position in memory via a label.
\paragraph{Bit Field}
\begin{center}
\begin{tabular}{| c | c | c | c | c | c | c | c | c | c | c | c | c | c | c | c |}
\hline
o&o&o&o&i&i&i&i&i&i&i&i&i&i&i&i\\ \hline
\end{tabular}
\end{center}

\paragraph{Example:}
\subparagraph{j label} jumps to label


\section{Other}
Special instructions such as pseudo instructions.

\begin{center}
\begin{tabular}{| c | c | c | c |}
\hline
Instruction & description & opcode & parameters \\ \hline
nop & No operation & 1111 & no parameters, all zeros\\ \hline
\end{tabular}
\end{center}

\subsection{Comments}
\subsubsection{nop}
\paragraph{}Effectively adds 0 to R0.

\section{Architecture}
\paragraph{}
This is a description of the blocks on the CPU that have already been designed, and how they work.
\paragraph{Note}
For writing VHDL we kept to some conventions. Input ports have the prefix \textbf{i}, output ports have the prefix \textbf{q} and signals have the prefix \textbf{l}. Also, for ease of typing, we kept the naming in lowercase.

\subsection{Top Level}
\paragraph{Description}
The way the FPGA interacts with the outside world between user and CPU.

For inputs, the CPU has a 100MHz clock, \textbf{clk}, and a couple of buttons and switches, \textbf{switches}, \textbf{buttons}, so that different modes can be implemented, like a bootloading mode, or a debugging mode. \textbf{rx} is a UART input.

For outputs, the CPU has the seven segment display and 8 LEDs. \textbf{tx} is a UART output.
\begin{minted}{vhdl}
entity booths_architecture is
  port( i_clk_100   : in  std_logic; --100MHz clock
        i_switches  : in  std_logic_vector(  1 downto 0 ); --two switches
        i_buttons   : in  std_logic_vector(  1 downto 0 ); --two buttons
        i_rx        : in  std_logic; --serial input of UART

        q_7_segment : out std_logic_vector(  6 downto 0 ); --7 segment display for debugging
        q_leds      : out std_logic_vector(  7 downto 0 ); --8 lines for LEDs
        q_tx        : out std_logic); --serial output of UART
end booths_architecture;
\end{minted}
\paragraph{Comments}
The operation of the other blocks with respect to the top level have not yet been implemented, as some are still being created and worked on. So far, there are two modes available: active mode and bootloading mode. a choice of input signal to choose the modes has yet to be decided, however most likely will be implemented as the position of the switches on power on. One button will be a reset signal that can be found in the code as \textbf{reset}.
\subsection{rx Communication}
\paragraph{Description}
UART input communication.

The rx communication protocol allows bootloading of the assembler to the FPGA. As can be seen from the entity below, it runs asynchronously. ie \textbf{clk} is not the same clock as the rest of the blocks. We get the \textbf{reset} and \textbf{baud 16} inputs from the switches on the FPGA board.
\begin{minted}{vhdl}
entity uart_rx is
  port( i_clk     : in  std_logic; --asynchronous clock
        i_reset   : in  std_logic; --reset signal
        i_baud_16 : in  std_logic; --16 times baud rate
        i_rx      : in  std_logic; --serial input line

        q_data    : out std_logic_vector( 7 downto 0 ); --byte to receive
        q_flag    : out std_logic); --toggle on byte receive
end uart_rx;
\end{minted}
When \textbf{reset} and \textbf{baud 16} are set, the bootloading sequence begins. We double clock \textbf{clk} after the start bit is received in order to read the other bits as shown in the image below:

\begin{figure}[h]
\includegraphics[scale=.65]{timing.png}
\end{figure}

\begin{minted}{vhdl}
	
\end{minted}
\end{document}
