\documentclass[a4paper]{article}

%% Language and font encodings
\usepackage[english]{babel}
\usepackage[utf8x]{inputenc}
\usepackage[T1]{fontenc}

%% Sets page size and margins
\usepackage[a4paper,top=3cm,bottom=2cm,left=3cm,right=3cm,marginparwidth=1.75cm]{geometry}

%% Useful packages
\usepackage{amsmath}
\usepackage{graphicx}
\usepackage[colorinlistoftodos]{todonotes}
\usepackage[colorlinks=true, allcolors=blue]{hyperref}

\title{Notes}
\author{Jacob Priddy, Andrew Glencross, Charles Oroko, Seth Ballance}

\begin{document}
\maketitle

\section{Processor}

\begin{itemize}
\item No Pipelining
\item No floating point
\item Registers only, no memory
\item Puts result of operation in the first operand
\item 4 bit opcode for 13 insturctions
\item 15 usable registers 1 PC register
\item 128 words of instruction memory or 128 instructions
\item Address registers are 4 bits: 0000 for the PC or 0001 for r1 so on and so forth
\item Words are 16 bits
\item All numbers are two's compliment
\end{itemize}

\section{Instruction Set}
\begin{itemize}
\item add
\item addi
\item shift left arithmetic
\item shift left logical
\item shift right
\item or
\item ori
\item and
\item andi
\item negate
\item jump zero
\item unconditional jump
\item load immediate
\item move
\item nop
\end{itemize}

\section{R-Type Instructions}

\begin{itemize}
\item 4 bit opcode
\item 4 bits for operand 1
\item 4 bits for operand 2
\end{itemize}



\begin{center}
\begin{tabular}{| c | c | c | c |}
\hline
Instruction & description & opcode & parameters \\ \hline
add & Add two registers & 0000 & 2 reg \\ \hline
and & And two registers & 0001 & 2 reg \\ \hline
or & Or two registers & 0010 & 2 reg \\ \hline
mov & Moves first register to second register & 0011 & 2 reg \\ \hline
\end{tabular}
\end{center}

\section{I-Type Instructions}
\begin{itemize}
\item 4 bit opcode
\item 4 bits for reg 1
\item 8 bits for immediate value
\end{itemize}

\begin{center}
\begin{tabular}{| c | c | c | c |}
\hline
Instruction & description & opcode & parameters \\ \hline
sll & shift left logical & 0100 & 1 reg 1 immediate \\ \hline
sla & shift left arithmetic & 0101 & 1 reg 1 immediate \\ \hline
sr & shift right arithmetic & 0110 & 1 reg 1 immediate \\ \hline
neg & Negate 1 register & 0111 & 1 reg \\ \hline
andi & Does the and operation on a register and an immedate & 1000 & 1 reg 1 immediate \\ \hline
addi & Add an immediate value to a register & 1001 & 1 reg 1 immediate \\ \hline
ori & Does the or operation on a register and an immedate & 1010 & 1 reg 1 immediate \\ \hline
loadi & Loads an immedate value into a register & 1011 & 1 reg 1 immedate \\ \hline
\end{tabular}
\end{center}

\section{J-Type}

\begin{itemize}
\item 4 bit opcode
\item 4 bits for comparison register
\item 8 bits for jump address
\end{itemize}

\begin{center}
\begin{tabular}{| c | c | c | c |}
\hline
Instruction & description & opcode & parameters \\ \hline
j & Jump unconditionally & 1100 & immediate \\ \hline
jz & Jump to an offset if a register is zero & 1101 & 1 reg 1 imm \\ \hline
\end{tabular}
\end{center}


\section{Other}
Special instructions such as pseudo instructions.


\begin{center}
\begin{tabular}{| c | c | c | c |}
\hline
Instruction & description & opcode & parameters \\ \hline
nop & No operation & 1111 & no parameters, all zeros\\ \hline
\end{tabular}
\end{center}

\end{document}

